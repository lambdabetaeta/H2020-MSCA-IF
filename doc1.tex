% !TEX root = ./IF-2017-Part_B.tex

% NB: Title page and table of contents are NO LONGER PART OF THE
% TEMPLATE (stated explicitly in applicant guide, missing from
% template).
%

\markStartPageLimit

\section{Excellence}
\label{sec:excellence}\footnote{Literature should be listed in
footnotes, font size 8 or 9.  All literature references will count
towards the page limit.}

\subsection{Quality and credibility of the research/innovation
project; level of novelty, appropriate consideration of
inter/multidisciplinary and gender aspects}
\label{sec:excellence_quality}

Provide an introduction, discuss the state-of-the-art, specific
objectives and give an overview of the action.

Discuss the research methodology and approach, highlighting the
type of research / innovation activities proposed.

Explain the originality and innovative aspects of the planned
research as well as the contribution that the action is expected
to make to advancements within the research field. Describe any
novel concepts, approaches or methods that will be implemented.

Discuss the interdisciplinary aspects of the action (if relevant).

Discuss the gender dimension in the research content (if
relevant).In research activities where human beings are involved
as subjects or end-users, gender differences may exist. In these
cases the gender dimension in the research content has to be
addressed as an integral part of the proposal to ensure the
highest level of scientific quality.


\subsection{Quality and appropriateness of the training and of the
two way transfer of knowledge between the researcher and the host}
\label{sec:excellence_transfer}

\noindent Outline how a two way transfer of knowledge will occur
between the researcher and the host institution(s):

\begin{itemize}
  \item
    Explain how the experienced researcher will gain new knowledge
    during the fellowship at the hosting organisation(s).
  \item
    Outline the previously acquired knowledge and skills that the
    researcher will transfer to the host organisation(s).

\end{itemize}

For \textbf{Global Fellowships} explain how the newly acquired
skills and knowledge in the Third Country will be transferred back
to the host institution in Europe (the beneficiary) during the
incoming phase.

Describe the training that will be offered. Typical
\textbf{training activities} in Individual Fellowships may
include:

\begin{itemize}
  \item 
    Primarily, training-through-research by the means of an
    \ul{individual personalised project}, under the guidance of the
    supervisor and other members of the research staff of the host
    organisation(s).
  \item
    Hands-on training activities for developing scientific skills (new
    techniques, instruments, research integrity, `big data'/`open
    science') and transferable skills (entrepreneurship, proposal
    preparation to request funding, patent applications, management of
    IPR, project management, task coordination, supervising and
    monitoring, take up and exploitation of research results)
  \item
    Inter-sectoral or interdisciplinary transfer of knowledge (e.g.
    through secondments)
  \item
    Participation in the research and financial management of the action
  \item
    Organisation of scientific/training/dissemination events
  \item
    Communication, outreach activities and horizontal skills
  \item
    Training dedicated to gender issues
\end{itemize}

\subsection{Quality of the supervision and of the integration in
the team/institution}
\label{sec:excellence_supervision}

Describe the qualifications and experience of the supervisor(s).
Provide information regarding the supervisors' level of experience
on the research topic proposed and their track record of work,
including main international collaborations, as well as the level
of experience in supervising/training especially at advanced level
(PhD, postdoctoral researchers). Information provided should
include participation in projects, publications, patents and any
other relevant results.

Describe the hosting arrangements.\footnote{The hosting
arrangements refer to the integration of the researcher to his new
environment in the premises of the host. It does not refer to the
infrastructure of the host as described in the Quality and
efficiency of the implementation criterion.} The application must
show that the experienced researcher will be well-integrated
within the team/institution so that all parties gain maximum
knowledge and skills from the fellowship. The nature and the
quality of the research group/environment as a whole should be
outlined, together with the measures taken to integrate the
researcher in the different areas of expertise, disciplines, and
international networking opportunities that the host could offer.

For \textbf{Global Fellowships} both phases should be described - for the outgoing phase, specify the practical arrangements in place to host a researcher coming from another country, and for the incoming phase specify the measures planned for the successful (re)integration of the researcher.


\subsection{Potential of the researcher to reach or re-enforce
professional maturity/independence during the fellowship}
\label{sec:excellence_maturity}

Researchers should \textbf{demonstrate} how their existing
professional experience, talents and the proposed research will
contribute to their development as independent/mature researchers,
during the fellowship. Explain the new competences and skills that
will be acquired and how they relate to the researcher’s existing
professional experience.

Please keep in mind that the fellowships will be awarded to the
most talented researchers as shown by the proposed research and
their track record (Curriculum Vitae, section 4), in relation to
their level of experience.

\section{Impact}
\label{sec:impact}

\subsection{Enhancing the future career prospects of the
researcher after the fellowship}
  \label{sec:impact_researcher}

Explain the expected impact of the planned research and training
(i.e. the added value of the fellowship) on the future career
prospects of the experienced researcher after the fellowship.
Focus on how the new competences and skills (as explained in 1.4)
can make the researcher more successful in their long-term career.

\subsection{Quality of the proposed measures to exploit and
disseminate the action results}
  \label{sec:impact_dissemination}

Describe how the new knowledge generated by the action will be
disseminated and exploited, and what the potential impact is
expected to be. Discuss the strategy for targeting peers
(scientific, industry and other actors, professional
organisations, policy makers, etc.) and to the wider community.
Also describe potential commercialisation, if applicable, and how
intellectual property rights will be dealt with, where relevant.

For more details refer to the
\href{http://ec.europa.eu/research/participants/docs/h2020-funding-guide/grants/grant-management/dissemination-of-results_en.htm}{``Dissemination
\& exploitation'' section of the H2020 Online Manual}

\medskip\noindent
Describe how the new knowledge generated by the action will be
disseminated and exploited, e.g. communicated, transferred into
other research settings or, if appropriate, commercialised.
Describe, when relevant, how intellectual property rights will be
dealt with.

\medskip\noindent
Concrete planning for section~\ref{sec:impact_dissemination} must
be included in the Gantt Chart (see
point~\ref{sec:implementation_work_plan}).


\subsection{Quality of the proposed measures to communicate the
project activities to different target audiences}
\label{sec:impact_communication}

Demonstrate how the planned public engagement activities
contribute to creating awareness of the performed research.
Demonstrate how both the research and results will be made known
to the public in such a way they can be understood by
non-specialists. 

The type of outreach activities could range from an Internet
presence, press articles and participating in European
Researchers' Night events to presenting science, research and
innovation activities to students from primary and secondary
schools or universities in order to develop their interest in
research careers. For more details, see the guide on
\href{http://ec.europa.eu/research/participants/data/ref/h2020/other/gm/h2020-guide-comm_en.pdf}{Communicating
EU research and innovation guidance for project participants} as
well as the
\href{http://ec.europa.eu/research/participants/docs/h2020-funding-guide/grants/grant-management/communication_en.htm}{``communication''
section of the H2020 Online Manual.}
\medskip\noindent
The frequency and nature of communication activities should be
outlined in the proposal. Concrete plans for the above must be
included as a deliverable.

\medskip\noindent
Concrete planning for communication activities must be included in the Gantt chart.

\section{Quality and Efficiency of the Implementation}
  \label{sec:implementation}

\subsection{Coherence and effectiveness of the work plan,
including appropriateness of the allocation of tasks and
resources} 
  \label{sec:implementation_work_plan}

Describe how the work planning and the resources mobilised will
ensure that the research and training objectives will be reached.
Explain why the number of person- months planned and requested for
the project is appropriate in relation to the proposed activities.

Additionally, a Gantt chart should be included in the text listing
the following:
\begin{itemize}
  \item Work Packages titles (there should be at least 1 WP); 
  \item List of major deliverables, if applicable;
  \item List of major milestones, if applicable;
  \item Secondments, if applicable.
\end{itemize}

\noindent
The schedule should be in terms of number of months elapsed from
the start of the action.

\begin{figure}[!htbp]
\begin{center}

\begin{ganttchart}[
    canvas/.append style={fill=none, draw=black!5, line width=.75pt},
    hgrid style/.style={draw=black!5, line width=.75pt},
    vgrid={*1{draw=black!5, line width=.75pt}},
    title/.style={draw=none, fill=none},
    title label font=\bfseries\footnotesize,
    title label node/.append style={below=7pt},
    include title in canvas=false,
    bar label font=\small\color{black!70},
    bar label node/.append style={left=2cm},
    bar/.append style={draw=none, fill=black!63},
    bar progress label font=\footnotesize\color{black!70},
    group left shift=0,
    group right shift=0,
    group height=.5,
    group peaks tip position=0,
    group label node/.append style={left=.6cm},
    group progress label font=\bfseries\small
  ]{1}{24}
  \gantttitle[
    title label node/.append style={below left=7pt and -3pt}
  ]{Month:\quad1}{1}
  \gantttitlelist{2,...,24}{1} \\
  \ganttgroup{Work Package}{1}{10} \\
  \ganttgroup{Deliverable}{5}{15} \\
  \ganttgroup{Milestone}{5}{5} \\
  \ganttgroup{Secondment}{20}{23} \\
  \ganttgroup{Conference}{16}{16} \\
  \ganttgroup{Workshop}{17}{17} \\
  \ganttgroup{Seminar}{18}{18} \\
  \ganttgroup{Dissemination}{23}{24} \\
  \ganttgroup{Public engagement}{4}{5} \\
  \ganttgroup{Other}{7}{10}
\end{ganttchart}

\end{center}
\caption{Example Gantt Chart}
\end{figure}

\subsection{Appropriateness of the management structure and
procedures, including risk management}
  \label{sec:implementation_management}

Describe the organisation and management structure, as well as the
progress monitoring mechanisms put in place, to ensure that
objectives are reached. Discuss the research and/or administrative
risks that might endanger reaching the action objectives and the
contingency plans to be put in place should risk occur.

If applicable, discuss any involvement of an entity with a capital
or legal link to the beneficiary (in particular, the name of the
entity, type of link with the beneficiary and tasks to be carried
out). 

If needed, please indicate here information on the support
services provided by the host institution (European offices, HR
services\dots).

\subsection{Appropriateness of the institutional environment
(infrastructure)}
\label{sec:implementation_infrastructure}

The active contribution of the beneficiary to the research and
training activities should be described. For Global Fellowships
the role of partner organisations in Third Countries for the
outgoing phase should also appear.


Give a description of the main tasks and commitments of the
beneficiary and all partner organisations (if applicable).

Describe the infrastructure, logistics, facilities offered insofar
as they are necessary for the good implementation of the action.

\markEndPageLimit
